\documentclass[a4paper]{exam}

\usepackage{amsmath}
\usepackage{amssymb}
\usepackage{array}
\usepackage{geometry}
\usepackage{hyperref}
\usepackage{titling}

\newcolumntype{C}{>{$}c<{$}} % math-mode version of "c" column type

\runningheader{CS/MATH 113}{WC04: Logical Inference}{\theauthor}
\runningheadrule
\runningfootrule
\runningfooter{}{Page \thepage\ of \numpages}{}

\printanswers

\title{Weekly Challenge 04: Logical Inference\\CS/MATH 113 Discrete Mathematics}
\author{team-name}  % <== for grading, replace with your team name, e.g. q1-team-420
\date{Habib University | Spring 2023}

\qformat{{\large\bf \thequestion. \thequestiontitle}\hfill}
\boxedpoints

\begin{document}
\maketitle

\begin{questions}

  \titledquestion{Sacred Secrets}[5] One of your helpful TAs has prepared the manual, ``Sacred Secrets: How to Stay Sane and Earn an A+''. However their own sanity is sadly no longer intact. The \LaTeX\ source and the repository got deleted and all that exists about the location of the only printed copy are the following instructions.
  \begin{enumerate}
  \item There is a hint at Learn Courtyard or at the Gym.
  \item If your TA is sitting in Ehsas or they are absent, then there is a hint at Learn Courtyard.
  \item If your TA is not sitting in Ehsaas, then there is a hint at the Gym.
  \item If there are people in Learn Courtyard, then there is no hint at Learn Courtyard.
  \item If there is a hint at Learn Courtyard, then the manual is at Zen Garden.
  \item If there is hint at the Gym, then the manual is at Earth Courtyard.
  \item If your TA is absent, then the manual is at Fire Courtyard.
  \end{enumerate}
  You notice that there are people in Learn Courtyard. Show how you can infer the location of the manual.

  \begin{solution}
    % Enter your solution here.
  \end{solution}

  \titledquestion{Logic Sauce}[5] In order to celebrate their performance in the recent quiz, a group of $n$ Discrete Mathematics students are hanging out at the Dhaba area. Rahim bhai asks them, ``Does everyone want fries?'' The first student replies, ``I do not know'', as does the second. In fact, the first $n-1$ students respond the same. The last student replies, ``No''. Rahim bhai then proceeds to prepare the order. What is the order and how did Rahim bhai deduce it?

  \begin{solution}
    % Enter your solution here.
  \end{solution}

  
\end{questions}

\end{document}



%%% Local Variables:
%%% mode: latex
%%% TeX-master: t
%%% End:
